\documentclass{jsarticle}

\usepackage{ascmac}
\usepackage{graphicx}
\usepackage[dvipdfmx]{color}
\usepackage{amssymb,amsmath,amsthm}
\usepackage{graphics}
\usepackage{fancybox, tcolorbox}
\tcbuselibrary{raster,skins, breakable}
\usepackage{nccmath}
\usepackage{tikz}
\usetikzlibrary{intersections, calc, cd}
\usepackage{bm}
\usepackage[italicdiff]{physics}
\usepackage{titlesec}
\usepackage{mathtools}
\usepackage{enumerate}
\usepackage{float}
\usepackage{datetime}
\usepackage[top=1cm, bottom=2cm, left=2cm, right=2cm]{geometry}

\newcommand{\xtd}{(\bm{x}', t')}
\newcommand{\xt}{(\bm{x}, t)}
\newcommand{\ave}[1]{\langle #1 \rangle}

\makeatletter
\newcommand*{\themonth}{\two@digits\month}
\newcommand*{\theday}{\two@digits\day}
\makeatother
\renewcommand{\today}{{\the\year}/{\themonth}/{\theday}}


\title{深層学習と統計神経力学 (甘利俊一) \\[2ex] \large Last Updated: \today, \quad writer: Shunsuke Yagi}
\author{}
\date{}

\begin{document}
\setcounter{section}{-1}
\maketitle
\section{序章}
\section{層状のランダム結合神経回路}
\begin{table}[H]
  \centering
  \begin{tabular}{|c|c|p{10 cm}|}
      \hline
      ページ数 & 該当箇所 & コメント \\ \hline
      16 & 一次元の出力信号 $z$ を出す & $z$ を $y$ に変更 \\ \hline
      17 & $w_i$ がガウス分布でなくても... & 分散の絶対値が有限である場合のみ中心極限定理によって $u$ はガウス分布に近づく。 \\ \hline
  \end{tabular}
  \label{tab:mem}
\end{table}

\end{document}